\documentclass[]{article}
\usepackage{lmodern}
\usepackage{amssymb,amsmath}
\usepackage{ifxetex,ifluatex}
\usepackage{fixltx2e} % provides \textsubscript
\ifnum 0\ifxetex 1\fi\ifluatex 1\fi=0 % if pdftex
  \usepackage[T1]{fontenc}
  \usepackage[utf8]{inputenc}
\else % if luatex or xelatex
  \ifxetex
    \usepackage{mathspec}
  \else
    \usepackage{fontspec}
  \fi
  \defaultfontfeatures{Ligatures=TeX,Scale=MatchLowercase}
\fi
% use upquote if available, for straight quotes in verbatim environments
\IfFileExists{upquote.sty}{\usepackage{upquote}}{}
% use microtype if available
\IfFileExists{microtype.sty}{%
\usepackage{microtype}
\UseMicrotypeSet[protrusion]{basicmath} % disable protrusion for tt fonts
}{}
\usepackage[margin=1in]{geometry}
\usepackage{hyperref}
\hypersetup{unicode=true,
            pdftitle={rmarkdown\_pdf},
            pdfauthor={Sébastien Renaut},
            pdfborder={0 0 0},
            breaklinks=true}
\urlstyle{same}  % don't use monospace font for urls
\usepackage{color}
\usepackage{fancyvrb}
\newcommand{\VerbBar}{|}
\newcommand{\VERB}{\Verb[commandchars=\\\{\}]}
\DefineVerbatimEnvironment{Highlighting}{Verbatim}{commandchars=\\\{\}}
% Add ',fontsize=\small' for more characters per line
\usepackage{framed}
\definecolor{shadecolor}{RGB}{248,248,248}
\newenvironment{Shaded}{\begin{snugshade}}{\end{snugshade}}
\newcommand{\AlertTok}[1]{\textcolor[rgb]{0.94,0.16,0.16}{#1}}
\newcommand{\AnnotationTok}[1]{\textcolor[rgb]{0.56,0.35,0.01}{\textbf{\textit{#1}}}}
\newcommand{\AttributeTok}[1]{\textcolor[rgb]{0.77,0.63,0.00}{#1}}
\newcommand{\BaseNTok}[1]{\textcolor[rgb]{0.00,0.00,0.81}{#1}}
\newcommand{\BuiltInTok}[1]{#1}
\newcommand{\CharTok}[1]{\textcolor[rgb]{0.31,0.60,0.02}{#1}}
\newcommand{\CommentTok}[1]{\textcolor[rgb]{0.56,0.35,0.01}{\textit{#1}}}
\newcommand{\CommentVarTok}[1]{\textcolor[rgb]{0.56,0.35,0.01}{\textbf{\textit{#1}}}}
\newcommand{\ConstantTok}[1]{\textcolor[rgb]{0.00,0.00,0.00}{#1}}
\newcommand{\ControlFlowTok}[1]{\textcolor[rgb]{0.13,0.29,0.53}{\textbf{#1}}}
\newcommand{\DataTypeTok}[1]{\textcolor[rgb]{0.13,0.29,0.53}{#1}}
\newcommand{\DecValTok}[1]{\textcolor[rgb]{0.00,0.00,0.81}{#1}}
\newcommand{\DocumentationTok}[1]{\textcolor[rgb]{0.56,0.35,0.01}{\textbf{\textit{#1}}}}
\newcommand{\ErrorTok}[1]{\textcolor[rgb]{0.64,0.00,0.00}{\textbf{#1}}}
\newcommand{\ExtensionTok}[1]{#1}
\newcommand{\FloatTok}[1]{\textcolor[rgb]{0.00,0.00,0.81}{#1}}
\newcommand{\FunctionTok}[1]{\textcolor[rgb]{0.00,0.00,0.00}{#1}}
\newcommand{\ImportTok}[1]{#1}
\newcommand{\InformationTok}[1]{\textcolor[rgb]{0.56,0.35,0.01}{\textbf{\textit{#1}}}}
\newcommand{\KeywordTok}[1]{\textcolor[rgb]{0.13,0.29,0.53}{\textbf{#1}}}
\newcommand{\NormalTok}[1]{#1}
\newcommand{\OperatorTok}[1]{\textcolor[rgb]{0.81,0.36,0.00}{\textbf{#1}}}
\newcommand{\OtherTok}[1]{\textcolor[rgb]{0.56,0.35,0.01}{#1}}
\newcommand{\PreprocessorTok}[1]{\textcolor[rgb]{0.56,0.35,0.01}{\textit{#1}}}
\newcommand{\RegionMarkerTok}[1]{#1}
\newcommand{\SpecialCharTok}[1]{\textcolor[rgb]{0.00,0.00,0.00}{#1}}
\newcommand{\SpecialStringTok}[1]{\textcolor[rgb]{0.31,0.60,0.02}{#1}}
\newcommand{\StringTok}[1]{\textcolor[rgb]{0.31,0.60,0.02}{#1}}
\newcommand{\VariableTok}[1]{\textcolor[rgb]{0.00,0.00,0.00}{#1}}
\newcommand{\VerbatimStringTok}[1]{\textcolor[rgb]{0.31,0.60,0.02}{#1}}
\newcommand{\WarningTok}[1]{\textcolor[rgb]{0.56,0.35,0.01}{\textbf{\textit{#1}}}}
\usepackage{graphicx,grffile}
\makeatletter
\def\maxwidth{\ifdim\Gin@nat@width>\linewidth\linewidth\else\Gin@nat@width\fi}
\def\maxheight{\ifdim\Gin@nat@height>\textheight\textheight\else\Gin@nat@height\fi}
\makeatother
% Scale images if necessary, so that they will not overflow the page
% margins by default, and it is still possible to overwrite the defaults
% using explicit options in \includegraphics[width, height, ...]{}
\setkeys{Gin}{width=\maxwidth,height=\maxheight,keepaspectratio}
\IfFileExists{parskip.sty}{%
\usepackage{parskip}
}{% else
\setlength{\parindent}{0pt}
\setlength{\parskip}{6pt plus 2pt minus 1pt}
}
\setlength{\emergencystretch}{3em}  % prevent overfull lines
\providecommand{\tightlist}{%
  \setlength{\itemsep}{0pt}\setlength{\parskip}{0pt}}
\setcounter{secnumdepth}{0}
% Redefines (sub)paragraphs to behave more like sections
\ifx\paragraph\undefined\else
\let\oldparagraph\paragraph
\renewcommand{\paragraph}[1]{\oldparagraph{#1}\mbox{}}
\fi
\ifx\subparagraph\undefined\else
\let\oldsubparagraph\subparagraph
\renewcommand{\subparagraph}[1]{\oldsubparagraph{#1}\mbox{}}
\fi

%%% Use protect on footnotes to avoid problems with footnotes in titles
\let\rmarkdownfootnote\footnote%
\def\footnote{\protect\rmarkdownfootnote}

%%% Change title format to be more compact
\usepackage{titling}

% Create subtitle command for use in maketitle
\newcommand{\subtitle}[1]{
  \posttitle{
    \begin{center}\large#1\end{center}
    }
}

\setlength{\droptitle}{-2em}

  \title{rmarkdown\_pdf}
    \pretitle{\vspace{\droptitle}\centering\huge}
  \posttitle{\par}
    \author{Sébastien Renaut}
    \preauthor{\centering\large\emph}
  \postauthor{\par}
      \predate{\centering\large\emph}
  \postdate{\par}
    \date{2018-09-06}


\begin{document}
\maketitle

{
\setcounter{tocdepth}{2}
\tableofcontents
}
\hypertarget{different-outputs}{%
\section*{Different outputs}\label{different-outputs}}
\addcontentsline{toc}{section}{Different outputs}

\begin{itemize}
\tightlist
\item
  There are five versions of this document. We will examine them:

  \begin{itemize}
  \tightlist
  \item
    \emph{.Rmd}: The \texttt{R\ markdown} document.
  \item
    \emph{.html}: A webpage as we saw in the previous section. Follow
    workshop using this version.
  \item
    \emph{.docx}: A MS Word document.
  \item
    \emph{.tex}: A \href{https://www.latex-project.org}{LaTeX} document.
  \item
    \emph{.pdf}: A Portable Document Format.
  \end{itemize}
\end{itemize}

\hypertarget{html-document}{%
\section{html document}\label{html-document}}

\begin{Shaded}
\begin{Highlighting}[]
\NormalTok{---    }
\NormalTok{title: "rmarkdown_pdf"    }
\NormalTok{author: "Sébastien Renaut"    }
\NormalTok{date: '2018-09-06'    }
\NormalTok{output: }
\NormalTok{  html_document: }
\NormalTok{    toc: yes }
\NormalTok{    theme: cerulean }
\NormalTok{--- }
\end{Highlighting}
\end{Shaded}

\begin{itemize}
\item
  You can specify it when you create a new \texttt{Rmarkdown} document.
\item
  You can also specify it later in the header.
\item
  Then, it's just a matter of kniting the document!
\end{itemize}

\hypertarget{microsoft-word}{%
\section{Microsoft Word}\label{microsoft-word}}

\begin{verbatim}
---  
title: "rmarkdown_docx"  
author: "Sébastien Renaut"  
date: '2018-09-06'  
output: 
  word_document: 
    toc: yes
---   
\end{verbatim}

\begin{itemize}
\item
  Little documentation, few options \& configurations are possible (This
  is probably not the format that should be promoted, as it moves away
  from an open source environment).
\item
  Can specify a \href{https://www.libreoffice.org/}{LibreOffice}
  OpenDocument Text (\texttt{output:\ odt\_document}) or Rich Text
  Format (\texttt{output:\ rtf\_document}) instead.
\item
  FYI, there is a spellchecker in \texttt{Rstudio}: Edit
  \textgreater{}Check Spelling\ldots{}
\end{itemize}

\hypertarget{portable-document-format-.pdf}{%
\section{Portable Document Format
(.pdf)}\label{portable-document-format-.pdf}}

\begin{verbatim}
---    
title: "rmarkdown_pdf"    
author: "Sébastien Renaut"    
date: '2018-09-06'    
output: 
  pdf_document:
    keep_tex: true
    toc: yes  
---    
\end{verbatim}

\begin{itemize}
\tightlist
\item
  You need an extra step to go from a LaTeX (\emph{.tex}) format to a
  \emph{.pdf}. This is handled by the \texttt{R\ tinytex\ pdflatex}
  function in R.
\end{itemize}

\includegraphics[width=5.20833in,height=\textheight]{../figures/pandoc1.png}

\begin{itemize}
\item
  \href{https://www.latex-project.org}{LaTeX software} is a high-quality
  typesetting system.
\item
  It is the \emph{de facto} standard for the communication and
  publication of scientific documents.
\item
  LaTeX is available as free software
  \href{https://www.latex-project.org/get/}{here}.
\item
  If interested, follow this discussion:
  \href{https://ubuntuforums.org/showthread.php?t=395863}{\emph{Why
  LaTeX is such a bloated system?}}
\item
  So\ldots{}\href{https://yihui.name/tinytex/r/}{\emph{TinyTeX}} is a
  custom LaTeX distribution that is small in size
  (\textasciitilde{}150MB) but functions well in most cases, especially
  for \texttt{R} users .
\item
  \texttt{tinytex}
  \includegraphics[width=0.26042in,height=\textheight]{../figures/tinytex.png}
  is an R studio package that installs \emph{TinyTeX}.
\item
  You should know have all the tools to generate your fully reproducible
  manuscripts in \texttt{R}!
\end{itemize}

\hypertarget{exercice-1}{%
\section{Exercice 1}\label{exercice-1}}

\begin{itemize}
\tightlist
\item
  If you haven't done so, install the \texttt{tinytex} R package from
  the console and run \texttt{install\_tinytex()}. It may take a few
  minutes to download and compile (\textasciitilde{}150MB).
\end{itemize}

\begin{verbatim}
install.packages("tinytex")  
library(tinytex)  
install_tinytex()  
\end{verbatim}

\begin{itemize}
\item
  Create a new document, compile it as \emph{.pdf}.

  \begin{itemize}
  \tightlist
  \item
    Add a Table of Content.
  \item
    Add a graphic.
  \end{itemize}
\item
  Now compile it as a Word document (\emph{.docx})
\item
  Add some reference by specifying the \texttt{csl:\ ../csl/peerj.csl}
  and \texttt{bibliography:\ ../biblio/test\_library.bib} in the header
\end{itemize}

\hypertarget{further-customization}{%
\section{Further customization}\label{further-customization}}

\hypertarget{rticles}{%
\subsection{\texorpdfstring{\texttt{rticles}}{rticles}}\label{rticles}}

\begin{itemize}
\item
  R packages \texttt{rticles} is a (potentially) useful package to
  format articles according to the specification of a journal.
\item
  But first, you need to install it in the R console.
\end{itemize}

\begin{verbatim}
install.packages("rticles")
\end{verbatim}

\begin{itemize}
\tightlist
\item
  Once installed, you can start a new R markdown document according to
  your journal of interest.
\end{itemize}

\includegraphics[width=5.20833in,height=\textheight]{../figures/getstarted.png}\\
\hspace*{0.333em}\\
\includegraphics[width=5.20833in,height=\textheight]{../figures/from_template.png}\\
\hspace*{0.333em}

\begin{itemize}
\item
  Right now, few templates available.
\item
  Some templates may be slower to render, depending on what \emph{LaTeX}
  package they depend on and need to be installed.
\end{itemize}

\hypertarget{latex-template-manuscript}{%
\subsection{LaTeX template:
manuscript}\label{latex-template-manuscript}}

\begin{itemize}
\item
  This allows further options in the \emph{.Rmd} file when going from
  \emph{.tex} file to \emph{.pdf}.
\item
  You can build your own \emph{.tex} template if you know LaTeX\ldots{}
\item
  But, there are also many templates available on the web that you can
  use.
\item
  Here is one I like for
  \href{https://github.com/svmiller/svm-r-markdown-templates/blob/master/svm-latex-ms.tex}{manuscripts}
  (Thanks \href{https://github.com/svmiller}{svmiller} on
  \includegraphics[width=0.35417in,height=\textheight]{../figures/octocat.png})

  \begin{itemize}
  \item
    For example, using this template, I am writing a scientific paper
    \href{https://github.com/seb951/Acadian_seaplants/blob/master/manuscript_Rmd/Acadian_seaplants_v5.pdf}{entirely
    in R markdown}.\\
    \hspace*{0.333em}\\
    \includegraphics[width=5.20833in,height=\textheight]{../figures/acadian.png}\\
    \hspace*{0.333em}\\
    \hspace*{0.333em}
  \item
    The only real objection I see, is integrating the comments of
    co-authors.
  \end{itemize}
\end{itemize}

\hypertarget{latex-template-cv}{%
\subsection{LaTeX template: CV}\label{latex-template-cv}}

\begin{itemize}
\item
  Here is a template I like for
  \href{https://github.com/svmiller/svm-r-markdown-templates/blob/master/svm-latex-cv.tex}{\emph{Curriculum
  Vitae}}

  \begin{itemize}
  \tightlist
  \item
    For example, using this template, I re-wrote my
    \href{http://sebastien.renaut.com/wp-content/uploads/2019/02/cv.pdf}{CV}
    to give it a fresh look!\\
    \hspace*{0.333em}
  \end{itemize}
\end{itemize}

\includegraphics[width=6.25in,height=\textheight]{../figures/CV.png} ~

\begin{itemize}
\tightlist
\item
  Let's briefly examine and knit the \emph{svm-rmarkdown-cv.Rmd} file in
  the reference\_material directory.
\end{itemize}

\hypertarget{presentations}{%
\section{Presentations}\label{presentations}}

\begin{verbatim}
---
title: "Untitled"
author: "Sebastien Renaut"
date: "27/02/2019"
output: ioslides_presentation
---
\end{verbatim}

\begin{itemize}
\tightlist
\item
  You can also generate Powerpoint-like presentations.
\end{itemize}

\includegraphics[width=6.25in,height=\textheight]{../figures/slides.png}

\hypertarget{bookdown}{%
\section{Bookdown}\label{bookdown}}

\begin{itemize}
\tightlist
\item
  \href{https://bookdown.org/}{Bookdown}
  \includegraphics[width=0.20833in,height=\textheight]{../figures/bookdown.png}
  is an open-source R package that facilitates writing books and
  long-form articles/reports with R Markdown.
\end{itemize}

\includegraphics[width=5.20833in,height=\textheight]{../figures/rmarkdown.png}


\end{document}
